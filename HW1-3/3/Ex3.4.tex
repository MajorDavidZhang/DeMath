\documentclass{article} % For LaTeX2e
\usepackage{nips15submit_e,times}
\usepackage[colorlinks,linkcolor=red]{hyperref}
\usepackage{url}
\usepackage{amsmath}
\usepackage{graphicx}
\usepackage{float}
\usepackage{bm}
\usepackage{amssymb}
%\documentstyle[nips14submit_09,times,art10]{article} % For LaTeX 2.09


\title{CS499 Homework 3 (First Draft)}


\author{
	Intersteller\thanks{ Use footnote for providing further information
		about author (webpage, alternative address)---\emph{not} for acknowledging
		funding agencies.}
	Department of Computer Science
	Cranberry-Lemon University
	Pittsburgh, PA 15213
}

% The \author macro works with any number of authors. There are two commands
% used to separate the names and addresses of multiple authors: \And and \AND.
%
% Using \And between authors leaves it to \LaTeX{} to determine where to break
% the lines. Using \AND forces a linebreak at that point. So, if \LaTeX{}
% puts 3 of 4 authors names on the first line, and the last on the second
% line, try using \AND instead of \And before the third author name.

\newcommand{\fix}{\marginpar{FIX}}
\newcommand{\new}{\marginpar{NEW}}

%\nipsfinalcopy % Uncomment for camera-ready version

\begin{document}
	\textbf{Exercise 3.4}\par
	\begin{enumerate}
	\item We have $Pr(T=n)=p(1-p)^{n-1}$,$n=1,2,\cdots$ Thus $T$ obeys geometric distribution.\par
	Then $$E(T^2)=\sum_{n=1}^\infty n^2p(1-p)^{n-1}$$.\par
	Here, we have 
	$$
	\sum_{n=1}^\infty nx^{n-1}=(\sum_{n=1}^\infty x^n)'=(\frac{x}{1-x})'=\frac{1}{(1-x)^2}
	$$
	and 
	$$
	\sum_{n=1}^\infty n^2x^{n-1}=(\sum_{n=1}^\infty nx^n)'=(x\sum_{n=1}^\infty nx^{n-1})'=(\frac{x}{(1-x)^2})'=\frac{1+x}{(1-x)^3}
	$$
	Then we order $x=1-p$, we have 
	$$
	E(T^2)=\sum_{n=1}^\infty n^2p(1-p)^{n-1}=p\sum_{n=1}^\infty n^2(1-p)^{n-1}=p\times \frac{2-p}{p^3}=\frac{2-p}{p^2}
	$$
	\item  We have 
	\begin{align*}
	E(\frac{1}{T})&=\sum_{n=1}^\infty \frac{1}{n} p(1-p)^{n-1}\\
	&=\frac{p}{1-p}\sum_{n=1}^\infty \frac{(1-p)^n}{n}\\
	&=\frac{p}{1-p}\sum_{n=1}^\infty \int_{1}^{p}-(1-p)^{n-1}\,dp\\
	&=\frac{p}{1-p}\int_{1}^{p}\{\sum_{n=1}^\infty -(1-p)^{n-1}\}\,dp\\
	&=\frac{p}{1-p}\int_{1}^{p}-\lim_{n\to\infty}\frac{1-(1-p)^{n}}{p}\,dp\\
	&=\frac{p}{1-p}\int_{1}^{p}-\frac{1}{p}\,dp\\
	&=-\frac{plnp}{1-p}
	\end{align*}
	\end{enumerate}
	
\end{document}

