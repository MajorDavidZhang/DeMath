\documentclass{article} % For LaTeX2e
\usepackage{nips15submit_e,times}
\usepackage[colorlinks,linkcolor=red]{hyperref}
\usepackage{url}
\usepackage{amsmath}
\usepackage{graphicx}
\usepackage{float}
\usepackage{bm}
\usepackage{amssymb}
%\documentstyle[nips14submit_09,times,art10]{article} % For LaTeX 2.09


\title{CS499 Homework 3 (First Draft)}


\author{
	Intersteller\thanks{ Use footnote for providing further information
		about author (webpage, alternative address)---\emph{not} for acknowledging
		funding agencies.}
	Department of Computer Science
	Cranberry-Lemon University
	Pittsburgh, PA 15213
}

% The \author macro works with any number of authors. There are two commands
% used to separate the names and addresses of multiple authors: \And and \AND.
%
% Using \And between authors leaves it to \LaTeX{} to determine where to break
% the lines. Using \AND forces a linebreak at that point. So, if \LaTeX{}
% puts 3 of 4 authors names on the first line, and the last on the second
% line, try using \AND instead of \And before the third author name.

\newcommand{\fix}{\marginpar{FIX}}
\newcommand{\new}{\marginpar{NEW}}

%\nipsfinalcopy % Uncomment for camera-ready version

\begin{document}
	
	
	\maketitle
	
	
	\textbf{Exercise 3.1}\par
    $1$.\\
    For $T_{10}=n$,there are $n-1$ situations for the first $n-2$ digits of the string:
    $$ \left\{
    \begin{aligned}
    0,0,0,\cdots,0,0 \\
    0,0,0,\cdots,0,1 \\
    \vdots\\
    1,1,1,\cdots,1,1
    \end{aligned}
    \right.
    $$
    Based on the probability calculation formula of classical probabilities,we have
    $$
    Pr\left[T_{10}=n\right]=\frac{n-1}{2^{n-2}}\cdot\frac{1}{4}=\frac{n-1}{2^n}
    $$
    $2$.\\
    Based on the knowledge about $A_n$ ($A_n$=\{$x$ $\in$ $(0,1)^n$ $|$ $x$ does not contain $11$\}),we have
    $$
    Pr\left[T_{11}=n\right]=\frac{|A_{n-2}|}{2^{n-2}}\cdot\frac{1}{4}=\frac{F_n}{2^n}
    $$
    
    $3$.\\
    Based on method $3$,we have
    $$ \left\{
    \begin{aligned}
    E(T_{10})=1+\frac{1}{2}\cdot E(T_{10})+\frac{1}{2}\cdot E(T_0) \\
    E(T_0)=1+\frac{1}{2}E(T_0)+\frac{1}{2}\cdot 0 \\
    \end{aligned}
    \right.
    $$
    So we have
    $$
    E(T_{10})=4.
    $$
    $$
    E(T_{11})=1+\frac{1}{2}\cdot E(T_{11})+\frac{1}{2}\cdot [1+\frac{1}{2}\cdot E(T_{11})]
    $$
    So we have
    $$
    E(T_{11})=6.
    $$



	



	\textbf{Exercise 2.2}\par






	
	



	
	\textbf{Exercise 2.3}\par
	





	\textbf{Exercise 2.4}\par
	
\end{document}

