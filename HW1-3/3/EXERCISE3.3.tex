\documentclass{article} % For LaTeX2e
%\usepackage{nips15submit_e,times}
\usepackage[colorlinks,linkcolor=red]{hyperref}
\usepackage{url}
\usepackage{amsmath}
\usepackage{graphicx}
\usepackage{float}
\usepackage{bm}
\usepackage{amssymb}
%\documentstyle[nips14submit_09,times,art10]{article} % For LaTeX 2.09


\title{CS499 Homework 3 (First Draft)}


\author{
	Intersteller\thanks{ Use footnote for providing further information
		about author (webpage, alternative address)---\emph{not} for acknowledging
		funding agencies.}
	Department of Computer Science
	Cranberry-Lemon University
	Pittsburgh, PA 15213
}

% The \author macro works with any number of authors. There are two commands
% used to separate the names and addresses of multiple authors: \And and \AND.
%
% Using \And between authors leaves it to \LaTeX{} to determine where to break
% the lines. Using \AND forces a linebreak at that point. So, if \LaTeX{}
% puts 3 of 4 authors names on the first line, and the last on the second
% line, try using \AND instead of \And before the third author name.

\newcommand{\fix}{\marginpar{FIX}}
\newcommand{\new}{\marginpar{NEW}}

%\nipsfinalcopy % Uncomment for camera-ready version

\begin{document}
	
	
	\maketitle
	
	\section{Fibonacci Numbers and Other Recurrences}
	\subsection{Identities among the Fibonacci Numbers}
	 \textbf{Exercise 3.1}\par
	


	\textbf{Exercise 3.2}\par
	
	\textbf{Exercise 3.3}\par
	
	\textbf{ 1.} there are 4 circumstances of last bit pair of $\{x\}$ and $\{y\}$ sequence, which are $\{0,0\},\{1,0\},\{0,1\} and \{1,1\}$.Assume the expectation of $T$ under each circumstance is $E_1,E_2,E_3,E_4$, we have
    $$
        \begin{cases}
        E_1=1+\frac{1}{4}E_1+\frac{1}{4}E_2+\frac{1}{4}E_3+\frac{1}{4}E_4\\
        E_2=1+\frac{1}{4}E_2+\frac{1}{4}E_4\\
        E_3=1+\frac{1}{4}E_1+\frac{1}{4}E_2\\
        E_4=1+\frac{1}{4}E_2\\
        \end{cases}
    $$
	by solving the equations above, we get
	\begin{equation*}
	\begin{split}
	E_1=\frac{384}{121}\\
	E_2=\frac{20}{11}\\
	E_3=\frac{272}{121}\\
	E_4=\frac{16}{11}
	\end{split}
	\end{equation*}
	so we get $E[T]$:\\
	$E[T]=\frac{1}{4}(E_1+E_2+E_3+E_4) = \frac{384}{121} $\\
	\textbf{ 2.}
	
	\textbf{(a)}similar to the solution above,Assume the prabability of the event "10 appears in x before 11 appears in y" under each circumstance is $Pra_1,Pra_2,Pra_3,Pra_4$, we have
	$$
	\begin{cases}
	Pra_1=\frac{1}{4}Pra_1+\frac{1}{4}Pra_2+\frac{1}{4}Pra_3+\frac{1}{4}Pra_4\\
	Pra_2=\frac{1}{2}+\frac{1}{4}Pra_2+\frac{1}{4}Pra_4\\
	Pra_3=\frac{1}{4}Pra_1+\frac{1}{4}Pra_2\\
	Pra_4=\frac{1}{4}+\frac{1}{4}Pra_2\\
	\end{cases}
	$$
	by solving the equations above, we get
	\begin{equation*}
	\begin{split}
	Pra_1=\frac{65}{121}\\
	Pra_2=\frac{41}{121}\\
	Pra_3=\frac{9}{11}\\
	Pra_4=\frac{5}{11}
	\end{split}
	\end{equation*}
	so we get $Pra$:\\
	$Pra=\frac{1}{4}(Pra_1+Pra_2+Pra_3+Pra_4) = \frac{65}{121} $\\
	
	\textbf{(b)}similar to the solution above,Assume the prabability of the event "10 appears in x at the same time with 11 appears in y" under each circumstance is $Prb_1,Prb_2,Prb_3,Prb_4$, we have
	$$
	\begin{cases}
	Prb_1=\frac{1}{4}Prb_1+\frac{1}{4}Prb_2+\frac{1}{4}Prb_3+\frac{1}{4}Prb_4\\
	Prb_2=\frac{1}{4}Prb_2+\frac{1}{4}Prb_4\\
	Prb_3=\frac{1}{4}Prb_1+\frac{1}{4}Prb_2\\
	Prb_4=\frac{1}{4}+\frac{1}{4}Prb_2\\
	\end{cases}
	$$
	by solving the equations above, we get
	\begin{equation*}
	\begin{split}
	Prb_1=\frac{17}{121}\\
	Prb_2=\frac{7}{121}\\
	Prb_3=\frac{1}{11}\\
	Prb_4=\frac{3}{11}
	\end{split}
	\end{equation*}
	so we get $Prb$:\\
	$Prb=\frac{1}{4}(Prb_1+Prb_2+Prb_3+Prb_4) = \frac{17}{121} $\\
	
	\textbf{(c)}similar to the solution above,Assume the prabability of the event "10 appears in x after 11 appears in y" under each circumstance is $Prc_1,Prc_2,Prc_3,Prc_4$, we have
	$$
	\begin{cases}
	Prc_1=\frac{1}{4}Prc_1+\frac{1}{4}Prc_2+\frac{1}{4}Prc_3+\frac{1}{4}Prc_4\\
	Prc_2=\frac{1}{4}Pra_2+\frac{1}{4}Pra_4\\
	Prc_3=\frac{1}{2}+\frac{1}{4}Prc_1+\frac{1}{4}Prc_2\\
	Prc_4=\frac{1}{4}+\frac{1}{4}Prc_2\\
	\end{cases}
	$$
	by solving the equations above, we get
	\begin{equation*}
	\begin{split}
	Prc_1=\frac{39}{121}\\
	Prc_2=\frac{73}{121}\\
	Prc_3=\frac{1}{11}\\
	Prc_4=\frac{3}{11}
	\end{split}
	\end{equation*}
	so we get $Prc$:\\
	$Prc=\frac{1}{4}(Prc_1+Prc_2+Prc_3+Prc_4) = \frac{39}{121} $\\




	\textbf{Exercise 3.4}\par






	
\end{document}

