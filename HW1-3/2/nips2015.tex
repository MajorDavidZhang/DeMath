\documentclass{article} % For LaTeX2e
\usepackage{nips15submit_e,times}
\usepackage[colorlinks,linkcolor=red]{hyperref}
\usepackage{url}
\usepackage{amsmath}
\usepackage{graphicx}
\usepackage{float}
\usepackage{bm}
\usepackage{amssymb}
%\documentstyle[nips14submit_09,times,art10]{article} % For LaTeX 2.09


\title{CS499 Homework 2 (First Draft)}


\author{
	Intersteller\thanks{ Use footnote for providing further information
		about author (webpage, alternative address)---\emph{not} for acknowledging
		funding agencies.}
	Department of Computer Science
	Cranberry-Lemon University
	Pittsburgh, PA 15213
}

% The \author macro works with any number of authors. There are two commands
% used to separate the names and addresses of multiple authors: \And and \AND.
%
% Using \And between authors leaves it to \LaTeX{} to determine where to break
% the lines. Using \AND forces a linebreak at that point. So, if \LaTeX{}
% puts 3 of 4 authors names on the first line, and the last on the second
% line, try using \AND instead of \And before the third author name.

\newcommand{\fix}{\marginpar{FIX}}
\newcommand{\new}{\marginpar{NEW}}

%\nipsfinalcopy % Uncomment for camera-ready version

\begin{document}
	
	
	\maketitle
	
	\section{Fibonacci Numbers and Other Recurrences}
	\subsection{Identities among the Fibonacci Numbers}
	\textbf{Exercise 2.1}\par
	



	\textbf{Exercise 2.2}\par






	
	



	
	\subsection{General Linear Recurrences}
	\textbf{Exercise 2.3}\par
	\textbf{solution 1:}\par
	For each eigenvalue of A, we have $|\lambda I-A|=0$, which is
	$$
 \begin{vmatrix}
   \lambda -a_1 & -a_2 & -a_3 & \cdots &-a_{k-1} & -a_k \\
   -1 & \lambda & 0 & \cdots & 0 & 0 \\
   0 & -1 & \lambda & \cdots & 0 & 0 \\
   \vdots & \vdots & \vdots & \ddots & \vdots & \vdots \\
   0 & 0 & 0 & \cdots & \lambda & 0\\
   0 & 0 & 0 & \cdots & -1 & \lambda\\
  \end{vmatrix}
$$

	Let 
	$$ row_{k-1} + \frac{1}{\lambda} row_k \rightarrow row_{k-1}$$
	$$row_{k-2} +\frac{1}{\lambda} row_{k-1} \rightarrow row_{k-2}$$
	$$\vdots$$
	$$row_1+\frac{1}{\lambda}row_2 \rightarrow row_1$$
	We get 
	$$
 \begin{vmatrix}
   \lambda -a_1-\frac{a_2}{\lambda}-\cdots - \frac{a_k}{\lambda ^{k-1}} & \cdots & \cdots & \cdots & \cdots  \\
   0 & \lambda & 0 & \cdots & \cdots \\
   0 & 0 & \lambda & \cdots & \cdots \\
   \vdots & \vdots & \vdots & \ddots & \vdots \\
   0 & 0 & 0 & \cdots & \lambda \\
  \end{vmatrix}=0
$$
$$\Rightarrow (\lambda -a_1-\frac{a_2}{\lambda}-\cdots - \frac{a_k}{\lambda ^{k-1}})\dot \lambda^{k-1}=0$$
$$\Rightarrow \lambda^{k}=a_1 \lambda^{k-1}+a_2 \lambda^{k-2}+\cdots + a_{k-1}\lambda +a_k $$

\textbf{solution 2:}\par
We suppose that for each enigenvalue $\lambda$, the enigenvector is
$$\begin{pmatrix}
\lambda^{k-1}\\
\lambda^{k-2}\\
\vdots\\
1
\end{pmatrix}
$$
Thus,$$
A\alpha = \begin{pmatrix}
a_1 \lambda^{k-1}+a_2\lambda^{k-2}+\cdots +a_k\\
\lambda^{k-1}\\
\vdots\\
\lambda
\end{pmatrix}
$$

$$\lambda \alpha = \begin{pmatrix}
\lambda^{k}\\
\lambda^{k-1}\\
\vdots\\
\lambda
\end{pmatrix}$$

According to the definition of enigenvalue, $A\alpha =\lambda \alpha$. So, if $\lambda^{k}=a_1 \lambda^{k-1}+a_2 \lambda^{k-2}+\cdots + a_{k-1}\lambda +a_k$  \textbf{(1)}, $\lambda$ is an eigenvalue of A, and 
$\alpha= \begin{pmatrix}
\lambda^{k-1}\\
\lambda^{k-2}\\
\vdots\\
1
\end{pmatrix}
$ is its corresponding eigenvector.\par
Also, n-dimension matrix has n eigenvalues, and equation \textbf{(1)} has n roots. So, $\lambda$ is an eigenvalue of A if and only if equation \textbf{(1)} is satisfied. 






	\textbf{Exercise 2.4}\par
	
\end{document}

