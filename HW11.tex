\documentclass{article} % For LaTeX2e
%\usepackage{nips15submit_e,times}
\usepackage[colorlinks,linkcolor=red]{hyperref}
\usepackage{url}
\usepackage{amsmath}
\usepackage{graphicx}
\usepackage{float}
\usepackage{bm}
\usepackage{amssymb}
%\documentstyle[nips14submit_09,times,art10]{article} % For LaTeX 2.09
\begin{document}
\textbf{Exercise 11.1}\par

	 As we have proved in \textbf{Exercise 8.8}, the largest antichain of $\{0,1\}^n$ is $\binom{n}{\lfloor n/2\rfloor}$. \par

	 We define a layer as a set of strings containing same number of $'1'$ and is sorted by how many  $'1'$ a string in this layer contains.\par

	 \textbf{1.}There are $\binom{n}{\lfloor n/2\rfloor}$ strings in the middle layer, which has the most strings. Since any two strings from the same layer are not comparable, there are at least $\binom{n}{\lfloor n/2\rfloor}$ chain partitions.\par

	 \textbf{2.} All strings in any layer except the middle one can form chains with unique strings in its adjacent layer with the following method:\par

	 Since $0\le k < n/2 $, there are more $'0'$ than $'1'$ in this layer, we calculate a strings score by the following rules: scan the string from the beginning and the initial score is $0$, add one if current digit is $0$,minus one otherwise. Find the digit where the first highest score appears (which must be a $'0'$),change it to $1$. Then we get a string belongs to its adjacent layer and these two strings can form a chain(they are comparable).Now we prove that this string is unique:\par

	 Assume that there are two different strings that transform into a same string. Assume that the first string changes the i-th digit, and the other changes the j-th digit (with no loss of generality, assume $i<j$).  Then the i-th digit of the second string and the j-th digit of the first string are $0$, whereas other digits are the same. Assume that the score of the $(i-1)$-th digit is k.Then the score of the i-th digit is $k+1$ for the first string and $(k-1)$ for the second. Assume that the score of the $(j-1)$-th digit for the first string is $k+1+p$,then the score of the $(j-1)$-th digit for the second string is $k-1+p$.The score of the j-th digit for the second string is $k+p$. Since the changing digit is where the first largest score occurs, we have

	 $$
	 k+1 \ge k+1+p
	 $$

	 $$
	 k<k+p
	 $$

	 where we get $p\leq 0$ and $p>0$ which contradict each other. So  different strings cannot transform into a same string by the method.\par
	 
	 Accordingly, we can get a matching of size $(^n _k)$ between the k-th layer and the $(k+1)$-th layer.
	 
\textbf{Exercise 11.5}\par	 
	In the i-th layer, there are $(^n _i)$ vertexes, for each vertex, we can form a path that ends at the $(n-i)$-th layer with the method we show in \textbf{Exercise 11.1}. When we reach a layer with more '1' than '0', we can still use this method to form a matching for the following reason. Since we begin at the i-th layer$(i<n/2)$, we have $(n-i)$ '0's and i '1's, so the score of the last digit equals $(n-2i)$, each time we reach the next layer, the max score minus 1 for the  digits before the changing digit remains unchanged. Totally we cross $(n-2I-1)$ layers from the i-th layer to $(n-i)$-th layer, so before we reach the $(n-i)$-th layer, the max score of every string in such path is always positive, which means the algorithm is feasible. Therefore, we can find $(^n _i)$ paths as requested.
	 
