\documentclass{article} % For LaTeX2e
\usepackage{nips15submit_e,times}
\usepackage[colorlinks,linkcolor=red]{hyperref}
\usepackage{url}
\usepackage{amsmath}
\usepackage{graphicx}
\usepackage{float}
\usepackage{bm}
\usepackage{amssymb}
%\documentstyle[nips14submit_09,times,art10]{article} % For LaTeX 2.09


\title{CS499 Homework 11 (First Draft)}


\author{
	Intersteller\thanks{ Use footnote for providing further information
		about author (webpage, alternative address)---\emph{not} for acknowledging
		funding agencies.}
	Department of Computer Science
	Cranberry-Lemon University
	Pittsburgh, PA 15213
}

% The \author macro works with any number of authors. There are two commands
% used to separate the names and addresses of multiple authors: \And and \AND.
%
% Using \And between authors leaves it to \LaTeX{} to determine where to break
% the lines. Using \AND forces a linebreak at that point. So, if \LaTeX{}
% puts 3 of 4 authors names on the first line, and the last on the second
% line, try using \AND instead of \And before the third author name.

\newcommand{\fix}{\marginpar{FIX}}
\newcommand{\new}{\marginpar{NEW}}

\newtheorem{theorem}{}

%\nipsfinalcopy % Uncomment for camera-ready version

\begin{document}
	
	
	\maketitle
\textbf{Exercise 11.1}\par

	 As we have proved in \textbf{Exercise 8.8}, the largest antichain of $\{0,1\}^n$ is $\binom{n}{\lfloor n/2\rfloor}$. \par

	 We define a layer as a set of strings containing same number of $'1'$ and is sorted by how many  $'1'$ a string in this layer contains.\par

	 \textbf{1.}There are $\binom{n}{\lfloor n/2\rfloor}$ strings in the middle layer, which has the most strings. Since any two strings from the same layer are not comparable, there are at least $\binom{n}{\lfloor n/2\rfloor}$ chain partitions.\par

	 \textbf{2.} All strings in any layer except the middle one can form chains with unique strings in its adjacent layer with the following method:\par

	 Since $0\le k < n/2 $, there are more $'0'$ than $'1'$ in this layer, we calculate a strings score by the following rules: scan the string from the beginning and the initial score is $0$, add one if current digit is $0$,minus one otherwise. Find the digit where the first highest score appears (which must be a $'0'$),change it to $1$. Then we get a string belongs to its adjacent layer and these two strings can form a chain(they are comparable).Now we prove that this string is unique:\par

	 Assume that there are two different strings that transform into a same string. Assume that the first string changes the i-th digit, and the other changes the j-th digit (with no loss of generality, assume $i<j$).  Then the i-th digit of the second string and the j-th digit of the first string are $0$, whereas other digits are the same. Assume that the score of the $(i-1)$-th digit is k.Then the score of the i-th digit is $k+1$ for the first string and $(k-1)$ for the second. Assume that the score of the $(j-1)$-th digit for the first string is $k+1+p$,then the score of the $(j-1)$-th digit for the second string is $k-1+p$.The score of the j-th digit for the second string is $k+p$. Since the changing digit is where the first largest score occurs, we have

	 $$
	 k+1 \ge k+1+p
	 $$

	 $$
	 k<k+p
	 $$

	 where we get $p\leq 0$ and $p>0$ which contradict each other. So  different strings cannot transform into a same string by the method.\par
	
	 Accordingly, we can get a matching of size $(^n _k)$ between the k-th layer and the $(k+1)$-th layer.\par

\textbf{Exercise 11.2}\par
	This case, $|\Gamma(A)|\ge |A|$ for every $A\subseteq L$. We use contradiction to prove it.
	Assume $|\Gamma(A)|<|A|$, $\exists A\subseteq L$. Since every vertex has degree $d$, consider $A$, we have $d\times |A|$ edges.
	Since $\frac{d\times|A|}{|\Gamma(A)|}>d$, there must exist a vertex in $R$ which degree is larger than $d$. It is a contradiction.
	Therefore, $|\Gamma(A)|\ge |A|$ for every $A\subseteq L$. According to course video, if $|\Gamma(A)|\ge |A|$ for every $A\subseteq L$,
	then (and only then) there exists a matching of size $|L|$. $G$ has a perfect matching.\par 	


\textbf{Exercise 11.3}\par
    Based on the mathematical induction, we have:\par
    $(1)$ When $d=1$, obviously the edges $E(G)$ can be partitioned into $d$ perfect matchings.\par
    $(2)$ When $d>1$, we suppose when $d=n$, the edges $E(G)$ can be partitioned into $d$ perfect matchings. When $d=n+1$, we can find a matching $M$ of size $|L|$ based on the $11.2$, now we delete $M$ from $E(G)$ and we get a new bipartite graph $G\prime$ whose every vertex has degree $n$. Based on the inductive assumption, the edge $E(G\prime)$ can be partitioned into n perfect matchings. We can add $M$ to $M_1,\cdots,M_n$ and get matchings $M_1,\cdots,M_{n+1} \subset E(G)$ such that $(1) M_i \cap M_j = \emptyset for 1\leq i<j\leq n+1$ and $(2) M_1\cup M_2 \cup\cdots\cup M_{n+1} = E(G)$.\par
    Therefore, if G is a $d$-regular bipartite graph, the edges $E(G)$ can be partitioned into $d$ perfect matchings.\par

\textbf{Exercise 11.4}\par
    $1.$ For every vertex $X \in V$, add $X\prime$ to constitute $V\prime$. If the capacity of $X$ is $c(X)$ in $G$, the capacity of $edge(X,X\prime)$ in $(G\prime)$ is $c(X).$ If $edge(X,Y)\in E$, $edge(X\prime,Y)\in E\prime$ and $c(X\prime,Y)$ is $\infty$.If the original flow is from s to t, then the flow in $G\prime$ is from s to $t\prime$.\par
    $2.$
    \begin{figure}[H]
  	\centering
  	\includegraphics[width=9cm]{11_4_2.png}
  	\caption{}
  	\label{}
  	\end{figure}


    $3.$ We can set the capacity of every vertex in $G$ as $1$ and transform $G = (V,E,c)$ into a network $G\prime = (V\prime,E\prime,c\prime)$ by the method we give in $11.4.1$. Then we use Edmonds-Karp algorithm on the network. If the maximum flow is k, we find $k$ paths $p_1,p_2,\cdots p_k$, each from $s$ to $t$, such that the paths are internally vertex disjoint.\par

 \textbf{Exercise 11.5}\par	
	In the i-th layer, there are $(^n _i)$ vertexes, for each vertex, we can form a path that ends at the $(n-i)$-th layer with the method we show in \textbf{Exercise 11.1}. When we reach a layer with more '1' than '0', we can still use this method to form a matching for the following reason. Since we begin at the i-th layer$(i<n/2)$, we have $(n-i)$ '0's and i '1's, so the score of the last digit equals $(n-2i)$, each time we reach the next layer, the max score minus 1 for the  digits before the changing digit remains unchanged. Totally we cross $(n-2i-1)$ layers from the i-th layer to $(n-i)$-th layer, so before we reach the $(n-i)$-th layer, the max score of every string in such path is always positive, which means the algorithm is feasible. Therefore, we can find $(^n _i)$ paths as requested.\par



\textbf{Exercise 11.6}\par
	1. $e$ is always full. $b$, $c$, $d$, $f$, $g$, $h$, $i$ is sometimes full. $a$ is never full.\par
	2. $e$ is always crossing. $a$, $b$, $c$, $d$, $f$, $g$, $h$, $i$ is never crossing.\par
	
\textbf{Exercise 11.7}\par
	It is impossible to be always full and never crossing.\par
	Proof: Consider an edge $e$ which is always full. We assume the max flow is $MAX$ and $f(e)=c(e)$. We delete this edge and the max flow of residual network is $MAX-c(e)$.
	There must exist minimum cut $MAX-c(e)$ in residual network. We add edge $e$ to this cut and will get minimum cut $MAX$ in original network. Therefore, edge $e$ can be crossing and  is impossible to be a never-crossing edge.\par
	It is impossible to be sometimes full and always crossing, sometimes full and sometimes crossing, never full and always crossing, never full and sometimes crossing.\par
	Proof: These four cases can be proved together. No matter whether it is sometimes full or never full, the edge $e$ can be not full. We assume the max flow is $MAX$ and
	$f(e)=m<c(e)$. If edge $e$ belongs to a minimum cut, delete it and the max flow of residual network is $MAX-m$. The minimum cut $MAX-m$ of the residual network is $MAX-m$. If edge $e$ can be crossing, then the minimum cut in the original network will be $MAX-m+c(e)>MAX$, which controdicts the max-flow-min-cut theorem, so it is not the minimun cut. Therefore edge $e$ is never crossing. \par
	It is possible to be sometimes full and never crossing, never full and never crossing.\par
	Example:\par
    \begin{figure}[H]
  	\centering
  	\includegraphics[scale=0.5]{1.png}
  	\caption{}
  	\label{}
  	\end{figure}
	
	In this network, $b$, $c$, $d$, $f$, $g$, $h$, $i$ is sometimes full and $a$ is never full. And $a$, $b$, $c$, $d$, $f$, $g$, $h$, $i$ are all never crossing.\par



\textbf{Exercise 11.8}\par
	The network is shown in the following figure.
	\begin{figure}[H]
  	\centering
  	\includegraphics[scale=0.5]{8-1.png}
  	\caption{}
  	\label{}
  	\end{figure}

 	The Max-IMCF is 1, shown in the following figure.
 	\begin{figure}[H]
  	\centering
  	\includegraphics[scale=0.5]{8-2.png}
  	\caption{}
  	\label{}
  	\end{figure}

  	The Max-MCF is 1.5, shown in the following figure.
  	\begin{figure}[H]
  	\centering
  	\includegraphics[scale=0.5]{8-3.png}
  	\caption{}
  	\label{}
  	\end{figure}


 \textbf{Exercise 11.9}\par
	The network and the demands are shown in the following figure.
	\begin{figure}[H]
  	\centering
  	\includegraphics[scale=0.5]{9.png}
  	\caption{}
  	\label{}
  	\end{figure}
  	According to \textbf{11.8}, the F-MCF problem of this network is feasible, while the F-IMCF problem of this network is not feasible.\par




 \textbf{Question}\par

  	When we considered Exercise 11.5, we used the method in 11.4, setting the capacity of each vertex as 1 and trying to find a maximum flow of size$\binom{n}{i}$. Of course, we added a source and a sink and made their capacity infinite. Then we tried to prove its minimum cut is $\binom{n}{i}$. It was simple to find a cut of size $\binom{n}{i}$ by just cutting all the vertices in the i-th layer. However, we failed to prove it was the minimum cut. So we changed our method and constructed all those paths. We would like to know how to prove this cut is minimum.
\end{document}
