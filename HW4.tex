\documentclass{article} % For LaTeX2e
\usepackage{nips15submit_e,times}
\usepackage[colorlinks,linkcolor=red]{hyperref}
\usepackage{url}
\usepackage{amsmath}
\usepackage{graphicx}
\usepackage{float}
\usepackage{bm}
\usepackage{amssymb}
%\documentstyle[nips14submit_09,times,art10]{article} % For LaTeX 2.09


\title{CS499 Homework 4 (First Draft)}


\author{
	Intersteller\thanks{ Use footnote for providing further information
		about author (webpage, alternative address)---\emph{not} for acknowledging
		funding agencies.}
	Department of Computer Science
	Cranberry-Lemon University
	Pittsburgh, PA 15213
}

% The \author macro works with any number of authors. There are two commands
% used to separate the names and addresses of multiple authors: \And and \AND.
%
% Using \And between authors leaves it to \LaTeX{} to determine where to break
% the lines. Using \AND forces a linebreak at that point. So, if \LaTeX{}
% puts 3 of 4 authors names on the first line, and the last on the second
% line, try using \AND instead of \And before the third author name.

\newcommand{\fix}{\marginpar{FIX}}
\newcommand{\new}{\marginpar{NEW}}

%\nipsfinalcopy % Uncomment for camera-ready version

\begin{document}
	\textbf{Exercise 4.2}\par
	\begin{enumerate}
	\item Since $$E(T)=\sum_{n=0}^\infty (2n+1)C_{n}p^{n+1}(1-p)^{n}$$
	and
	\begin{align*}
	\sum_{n=0}^\infty (2n+1)C_{n}p^{n+1}(1-p)^{n}&=p\sum_{n=0}^\infty \frac{2n+1}{n+1} \binom{2n}{n}[p(1-p)]^{n}\\
	&<2p\sum_{n=0}^\infty \binom{2n}{n}[p(1-p)]^{n}\\
	&<2p\sum_{n=0}^\infty 2^{2n}[p(1-p)]^{n}\\
	&=2p\sum_{n=0}^\infty [4p(1-p)]^{n}\\
	\end{align*}
	we have $$E(T)<2p\sum_{n=0}^\infty [4p(1-p)]^{n}$$
	Since $p>\frac{1}{2}$, then $4p(1-p)<1$.\par
	Thus, $$E(T)<\frac{2p}{1-4p(1-p)}$$ $E(T)$ is finite.
	\item We denote $$g(x)=C_0+C_1x+C_2x^2+...+C_{n}x^{n}+...$$ where $C_{n}$ is Calatan number.\par
	We can compute that $$[g(x)]^2=C_0^2+(C_0C_1+C_1C_0)x+(C_0C_2+C_1^2+C_2C_0)x^2+...+(C_0C_{n}+C_1C_{n-1}+...+C_{n}C_0)x^{n}+...$$
	Since $$C_{n}=C_0C_{n-1}+C_1C_{n-2}+...+C_{n-1}C_0$$
	we have $$[g(x)]^2=C_0^2+C_2x+C_3x^2+C_4x^3+...+C_{n+1}x^{n}+...$$
	Since $C_0=C_1=1$, we have $x[g(x)]^2=g(x)-1$.\par
	We solve the equation and get
	$$g(x)=\frac{1\pm\sqrt{1-4x}}{2x}$$
	Since $g(0)=1$, thus 
	$$g(x)=\frac{1-\sqrt{1-4x}}{2x}$$
	$$g'(x)=\frac{1}{x\sqrt{1-4x}}-\frac{1-\sqrt{1-4x}}{2x^2}$$
	So,
	\begin{align*}
	E(T)&=\sum_{n=0}^\infty (2n+1)C_{n}p^{n+1}(1-p)^{n}\\
	&=2p\sum_{n=0}^\infty nC_{n}[p(1-p)]^{n}+p\sum_{n=0}^\infty C_{n}[p(1-p)]^{n}\\
	&=2p^2(1-p)\times (g[p(1-p)])'+p\times g[p(1-p)]\\
	&=\frac{2p^2(1-p)}{p(1-p)\sqrt{1-4p(1-p)}}-\frac{1-\sqrt{1-4p(1-p)}}{1-p}+p\times \frac{1-\sqrt{1-4p(1-p)}}{2p(1-p)}\\
	&=\frac{2p}{\sqrt{1-4p(1-p)}}-\frac{1-\sqrt{1-4p(1-p)}}{2(1-p)}\\
	&=\frac{2p}{2p-1}-\frac{2(1-p)}{2(1-p)}\\
	&=\frac{1}{2p-1}
	\end{align*}
	\end{enumerate}
	
	\textbf{Exercise 4.4}\par
	$$E[\frac{1}{T+1}]=\sum_{n=0}^\infty \frac{1}{2(n+1)}C_{n}p^{n+1}(1-p)^{n}$$
	We denote $$g(x)=C_0+C_1x+C_2x^2+...+C_{n}x^{n}+...$$ where $C_{n}$ is Calatan number.\par
	We can compute that $$[g(x)]^2=C_0^2+(C_0C_1+C_1C_0)x+(C_0C_2+C_1^2+C_2C_0)x^2+...+(C_0C_{n}+C_1C_{n-1}+...+C_{n}C_0)x^{n}+...$$
	Since $$C_{n}=C_0C_{n-1}+C_1C_{n-2}+...+C_{n-1}C_0$$
	we have $$[g(x)]^2=C_0^2+C_2x+C_3x^2+C_4x^3+...+C_{n+1}x^{n}+...$$
	Since $C_0=C_1=1$, we have $x[g(x)]^2=g(x)-1$.\par
	We solve the equation and get
	$$g(x)=\frac{1\pm\sqrt{1-4x}}{2x}$$
	Since $g(0)=1$, thus 
	$$g(x)=\frac{1-\sqrt{1-4x}}{2x}$$
	$$G(x)=\sum_{n=0}^\infty \frac{1}{n+1}C_{n}x^{n+1}=\int_{0}^{x} g(t)dt=\ln(\sqrt{1-4x}+1)-\sqrt{1-4x}-\ln 2+1$$
	Since $p=\frac{1}{2}$, then 
	\begin{align*}
	E[\frac{1}{T+1}]&=\frac{1}{4}\sum_{n=0}^\infty \frac{1}{n+1}C_{n}(\frac{1}{4})^n\\
	&=\sum_{n=0}^\infty \frac{1}{n+1}C_{n}(\frac{1}{4})^{n+1}\\
	&=G(\frac{1}{4})\\
	&=1-\ln 2
	\end{align*}

    \textbf{Exercise 4.5}\par
    $Proof$.
    Suppose $A_i$ whose strength is $a_i$ and $B_j$ whose strength is $b_j$ fight.After fighting,for $A_i$ we have
    $$
    E(strength_{A_i})=\frac{a_i}{a_i+b_j}\cdot (a_i+b_j)+\frac{b_j}{a_i+b_j}\cdot 0=a_i
    $$
    So after $\forall k$ wars,
    $$
    E(\sum strength_{A_i})=\sum_{i=1}^{m}a_i
    $$
    $$
    E(\sum strength_{B_j})=\sum_{j=1}^{n}b_j
    $$
    Now let's compute $E(\sum strength_{A_i})$ in a different way:\par
    Wars has only two results:Alice'team wins or Bob's team wins.\par
    The former happens with probability $p$,
    the latter with probability $1-p$.\par
    Thus
    $$
    E(\sum strength_{A_i})=(\sum_{i=1}^{m}a_i+\sum_{j=1}^{n}b_j)\cdot p+0\cdot (1-p)
    $$
    Therefore,
    $$
    p=\frac{\sum_{i=1}^{m}a_i}{\sum_{i=1}^{m}a_i+\sum_{j=1}^{n}b_j}
    $$
    So the probability of Alice's team winning does not depend on the order in which Alice and Bob send their monsters into the arena.


\end{document}
	
	
	\maketitle
