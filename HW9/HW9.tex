\documentclass{article} % For LaTeX2e
\usepackage{nips15submit_e,times}
\usepackage[colorlinks,linkcolor=red]{hyperref}
\usepackage{url}
\usepackage{amsmath}
\usepackage{graphicx}
\usepackage{float}
\usepackage{bm}
\usepackage{amssymb}
%\documentstyle[nips14submit_09,times,art10]{article} % For LaTeX 2.09


\title{CS499 Homework 9 (First Draft)}


\author{
	Intersteller\thanks{ Use footnote for providing further information
		about author (webpage, alternative address)---\emph{not} for acknowledging
		funding agencies.}
	Department of Computer Science
	Cranberry-Lemon University
	Pittsburgh, PA 15213
}

% The \author macro works with any number of authors. There are two commands
% used to separate the names and addresses of multiple authors: \And and \AND.
%
% Using \And between authors leaves it to \LaTeX{} to determine where to break
% the lines. Using \AND forces a linebreak at that point. So, if \LaTeX{}
% puts 3 of 4 authors names on the first line, and the last on the second
% line, try using \AND instead of \And before the third author name.

\newcommand{\fix}{\marginpar{FIX}}
\newcommand{\new}{\marginpar{NEW}}

\newtheorem{theorem}{}

%\nipsfinalcopy % Uncomment for camera-ready version

\begin{document}
	
	
	\maketitle
	
	
	\textbf{Exercise 10.5}\par
    Since dist$(s,t)=k$, we have a path $s\rightarrow u_1\rightarrow u_2 \cdots u_{k-1}\rightarrow t$.For convenience, we call $s:u_0,t:u_k$. We construct $V_0, V_1\cdots V_K$ as follows:\par
    Step$1$:$u_0\in V_0, u_1\in V_1, u_2\in V_2\cdots u_{k-1}\in V_{K-1}, u_k\in V_K.$\par
    Step$2$:$\forall v\in V \backslash V_k$, if there is an edge $(v,t), v\in V_{k-1}$. Obviously, $\forall n, 0\leq n \leq k-2, u_n$ can't link to t. Otherwise, we can skip $u_{k-1}$ and get a shorter path from $s$ to $t$, which contradicts $dist(s,t)=k$.\par
    Step$3$:$forall v\in V \backslash (V_k\cup V_{k-1})$, if $\exists w \in V_{k-1}, \exists edge (v,w), v\in V_{k-2}.$ Similarly, $\forall n, 0\leq n\leq k-3,$ $u_n$ can't link to any vertex in $V_{k-1}.$\par
    \vdots
    Step$k$:$\forall v\in V \backslash \bigcup_{i=2}^kV_i$, if $\exists w\in V_2, \exists edge (v,w), v\in V_1.$ Similarly, $u_0$ can't link to any vertex in $V_2$.\par
    Step$k+1$:$\forall v \in V $ and $ v \notin V_i, i \neq 1,2,3\cdots k, v \in V_0$.\par
    Therefore, if dist$(s,t)=k$, $(G,s,t,c)$ has a k-layering.


    \textbf{Exercise 10.7}\par
    Because there is a path from $s$ to $t$, we suppose $dist_G(s,t)=k$. Based on the construction method in $10.5$, we can find a k-layering and it is an optimal layering. Therefore, every network$(G,s,t,c)$ has an optimal layering, provided there is a path from $s$ to $t$.
	
  \textbf{Exercise 9.9}\par
	  $\forall S\in {\{0,1\}}^{\mathbb{N}}$, define $f(S)=T$ as follows.\par
 $$t_n=\sum_{i=1}^n(s_i+1)3^{i-1}$$.
Obviously, $f$ is a bijection. Let $X=\{f(S)|\forall S\in {\{0,1\}}^{\mathbb{N}}\}$. $X\cong {\{0,1\}}^{\mathbb{N}}$ is uncountable. And $f(S)$ is infinite. Whenever distinct $x,y \in X(x=\{x_1,x_2,...\})$, suppose $m=f^{-1}(x),n=f^{-1}(y)$ and assume the first different digit between $m$ and $n$ is the $k_{th}$ digit. Then $x\cap y=\{x_1,x_2,x_3,\cdots,x_{k-1}\}$, which is finite.
	
		\textbf{Question}\par
	 How to prove that ${\mathbb{R}}$ is smaller than $2^{\mathbb{R}}$.
	
\end{document}

