\documentclass{article} % For LaTeX2e
\usepackage{nips15submit_e,times}
\usepackage[colorlinks,linkcolor=red]{hyperref}
\usepackage{url}
\usepackage{amsmath}
\usepackage{graphicx}
\usepackage{float}
\usepackage{bm}
\usepackage{amssymb}
%\documentstyle[nips14submit_09,times,art10]{article} % For LaTeX 2.09


\title{CS499 Homework 10 (First Draft)}


\author{
	Intersteller\thanks{ Use footnote for providing further information
		about author (webpage, alternative address)---\emph{not} for acknowledging
		funding agencies.}
	Department of Computer Science
	Cranberry-Lemon University
	Pittsburgh, PA 15213
}

% The \author macro works with any number of authors. There are two commands
% used to separate the names and addresses of multiple authors: \And and \AND.
%
% Using \And between authors leaves it to \LaTeX{} to determine where to break
% the lines. Using \AND forces a linebreak at that point. So, if \LaTeX{}
% puts 3 of 4 authors names on the first line, and the last on the second
% line, try using \AND instead of \And before the third author name.

\newcommand{\fix}{\marginpar{FIX}}
\newcommand{\new}{\marginpar{NEW}}

\newtheorem{theorem}{}

%\nipsfinalcopy % Uncomment for camera-ready version

\begin{document}
	
	
	\maketitle
	
	
	\textbf{Exercise 10.1}\par
	Since $$\sum_{v\in V}f(s,v)=\sum_{v\in V\backslash S}f(s,v)+\sum_{v\in S}f(s,v)$$ we only need to prove that $$\sum_{v\in S}f(s,v)=\sum_{u\in S-s,\ v\in V\backslash S}f(u,v)$$\\
	Since $$\sum_{v\in S}f(s,v)=-\sum_{v\in S-s}f(v,s)$$ we only need to prove that $$\sum_{u\in S-s,\ v\in s+V\backslash S}f(u,v)=0$$
	It is obvious to see that $$\sum_{u\in S-s,\ v\in S-s}f(u,v)=0$$
	So, we only need to prove that $$\sum_{u\in S-s,\ v\in s+V\backslash S}f(u,v)+\sum_{u\in S-s,\ v\in S-s}f(u,v)=\sum_{u\in S-s,\ v\in V}f(u,v)=0$$
	According to the defination, $$\sum_{u\in S-s,\ v\in V}f(u,v)=0$$
	Done.


	\textbf{Exercise 10.2}\par
	Define the minimum cut between i and j as minCut(i,j). According to the Max Flow Min Cut Theorem, minCut(s,r)$\geq$k, minCut(r,t)$\geq$k. Obviously, minCut(s,t)$\geq$min\{minCut(s,r),minCut(r,t)\}$\geq$k, which means there is a flow from s to r of value k.

	\textbf{Exercise 10.3}\par
	Suppose there is a s-t-path in G that has less that k edges. Then, at least one edge in the path moves more than one level forward, which contradicts Definition 10.3. So, $dist(s,t)\geq k$.

	
	\textbf{Exercise 10.9}\par
	\textbf{proof} According to \textbf{Exercise 10.8}, a particular layering is no more optimal after at moat $m$ iterations. Since a layering is at least 1-layering and at most n-latering, after at most $m*n$ iterations, there is no optimal layering, which means there is no s-t-path, the algorithm terminates.
	
	
	\textbf{Exercise 10.10}\par
	\textbf{proof} According to \textbf{Exercise 10.9},  the Edmonds-Karp algorithm terminates after n·m iterations of the while-loop, which is to say, we can get the max flow $f$ after finite steps by Edmonds-Karp algorithm.
   
		\textbf{Question}\par
	 
	
\end{document}

