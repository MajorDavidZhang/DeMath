\documentclass{article} % For LaTeX2e
\usepackage{nips15submit_e,times}
\usepackage[colorlinks,linkcolor=red]{hyperref}
\usepackage{url}
\usepackage{amsmath}
\usepackage{graphicx}
\usepackage{float}
\usepackage{bm}
\usepackage{amssymb}
%\documentstyle[nips14submit_09,times,art10]{article} % For LaTeX 2.09


\title{CS499 Homework 6 (First Draft)}


\author{
	Intersteller\thanks{ Use footnote for providing further information
		about author (webpage, alternative address)---\emph{not} for acknowledging
		funding agencies.}
	Department of Computer Science
	Cranberry-Lemon University
	Pittsburgh, PA 15213
}

% The \author macro works with any number of authors. There are two commands
% used to separate the names and addresses of multiple authors: \And and \AND.
%
% Using \And between authors leaves it to \LaTeX{} to determine where to break
% the lines. Using \AND forces a linebreak at that point. So, if \LaTeX{}
% puts 3 of 4 authors names on the first line, and the last on the second
% line, try using \AND instead of \And before the third author name.

\newcommand{\fix}{\marginpar{FIX}}
\newcommand{\new}{\marginpar{NEW}}

%\nipsfinalcopy % Uncomment for camera-ready version

\begin{document}

	\maketitle
	\textbf{Exercise 6.1}\par

    $(1)$As the picture shows:\par
    \begin{figure}[H]
		\centering
		\includegraphics[scale=0.3]{p31.jpg}
		\caption{}
		\label{fig:1}
	\end{figure}
	The number of automorphisms is 2.
    $(2)$Suppose the original picture is\par
    \begin{figure}[H]
		\centering
		\includegraphics[scale=0.4]{p32.jpg}
		\caption{}
		\label{fig:2}
	\end{figure}
    Original $A$ position now is $A'$,so original $B$ position now is the point connected with $A'$ before and there are two possibilities.\par
    If $B'$ is determined,the original $F$ position is the other point connected with $A'$ apart from $B'$.\par
    Therefore other positions is also determined for the same reason, as in the following figure.\par
    \begin{figure}[H]
		\centering
		\includegraphics[scale=0.4]{p33.jpg}
		\caption{}
		\label{fig:3}
	\end{figure}
    So the number of automorphisms is $6 \times 2 = 12.$\par
    $(3)$For the same reason of $(2)$,if there are 3 points  determined , the graph is determined.\par
    So the number of automorphisms is $8 \times 3! = 48$.

	
	

	\textbf{Exercise 6.2}\par
    The complete graph on $999$ vertices has $C_{999}^{2}$ edges.The $|E|$ is odd,so the $|E|$ of original graph and the $|E|$ of complement graph is not equal.
    So there is no self-complementary graph on $999$ vertices.

	\textbf{Exercise 6.3}\par
	\textbf{Theorem:}  There is a self-complementary graph on $n$ vertices if and only if $n=4k$ or $n=4k+1$. (here and in the following, $k=1,2,3,\cdots$)\par
	\textbf{proof:}\\
	(1)  If $n=4k+2$ or $n=4k+3$, since a complete graph has $\frac{(|V|-1)(|V|-2)}{2}$ edges, $|E|$ will be an odd number. It is obvious that if $|E|$ is an odd number, the graph can not be self-complementary.\par
	(2)  If $n=4k$, we can show the graph is self-complementary in the following method:\\
	We divide the vertices into 4 sequences A, B, C and D, each sequence has k verticies. Then we add $\frac{(k-1)(k-2)}{2}$ edges into sequence A and sequence B respectively.After that we add $\frac{(k)(k-1)}{2}$ edges between sequence A and sequence B, sequence A and C, sequence B and sequence D respectively, as in the following figure. 
	\begin{figure}[H]
		\centering
		\includegraphics[scale=0.8]{p1.png}
		\caption{}
		\label{fig:4}
	\end{figure}

	In this way we get a graph G, whose complementary graph is G', which has $\frac{(k-1)(k-2)}{2}$ edges in sequence C and in sequence D respectively, and $\frac{(k)(k-1)}{2}$ edges between sequence A and sequence D, sequence B and sequence C, sequence C and sequence D respectively. Through the bijection $C\rightarrow A$ ($c_1\rightarrow a_1, c_2\rightarrow a_2,\cdots$ (and the same in the following)), $A\rightarrow D$, $B\rightarrow C$ and $D\rightarrow B$, we can show that G and G' are automorphism, which means G or G' is a self-complementary graph.\par
	(3)  If $n=4k+1$, we divide the vertices into 4 sequences same as above and 1 special dot. Then for the sequences we use the same method to add edges, except in graph G we add k edges between A and the dot, B and the dot respectively. Then we will have k edges between C and the dot, D and the dot respectively in G'. In the bijection we add $dot \rightarrow dot$. In this way, G or G' is still self-complementary graph.


	\textbf{Exercise 6.4}\par
	For $k=3$ and $k=4$, the corresponding graphs are in the following figure.
	\begin{figure}[H]
		\centering
		\includegraphics[scale=0.8]{p2.png}
		\caption{}
		\label{fig:5}
	\end{figure}
	\textbf{One general construction method:} For $k=n$, first we draw a polygon that has n edges. Then we add one vertex in the middle of the polygon and link it with n vertices respectively. After that we add 2 verticies on each edge. Finally, we add one vertex and one edge in every 3 vertices clockwise. 

	\textbf{Question:}\par
	In Exercise 6.7, if we do not include the constraint that P is a prime number, what conclusion can we get? First, we make a prime factorization on P, which is $P=P_1^{A_1}\cdot P_2^{A_2}\cdot \cdots \cdot P_n^{A_n}$
  	For every i, $P_i$ is prime number and $A_i$ is positive integer.
	We have a hypothesis that the number of vertices might be $O(P_{max}+A_{max})$. But we don’t know how to prove it. Or do we have another answer?


	\textbf{Exercise 6.5}\par
	For every $n\le6$, there is an asymmetric graph on $n$ vertices, like Figure $1$.
  	\begin{figure}[H]
  	\centering
  	\includegraphics[scale=0.4]{6.5.png}
  	\caption{}
  	\label{fig:1}
  	\end{figure}
	When $n=6$, the left graph is correct. When $n>6$, we just need to construct trees like right graphs. 
	There are only one vertex that has three degrees and  $n-4$ vertices are at the right of this vertex.
	 
	\textbf{Exercise 6.6}\par
	For graphs in \textbf{Exercise 6.5}, we just need to add cherries for every vertex, like Figure $2$.
	\begin{figure}[H]
  	\centering
  	\includegraphics[scale=0.4]{6.6.png}
  	\caption{}
  	\label{fig:1}
  	\end{figure}
	The graph has $3n$ vertices with $2^{n}$ automorphisms.
	\textbf{Exercise 6.7}\par
	\textbf{1.} We can use $3p+1$ to form such graph with the method showed in \textbf{Exercise 6.4}\par
	\textbf{2.} There is no such graph with less than $p$ verticles. Consider a graph with $p$ automorphisms that has as less verticles as possible.\par 
	If it has two unconnected parts, there are 2 cases: \textbf{1)} one of them has only 1 automorphisms, in which case we can delete this part for less verticles. \textbf{2)} one of them has $m$  automorphisms, the other has $n$ automorphisms, if these 2 parts are different, $p=m \times n$, otherwise $p=2\times m\times n$, which contradicts the assumption that $p$ is a prime number. So the graph must be connected.\par 
	Assume more than one automorphis bijection is $f$, if there exits a verticle $a$ such that $f(a)=a$ and there are $m$ such automorphis, for an automorphis bijection $g$ such that $g(a) \not=a$, we can form a new bijection $t=f \dot g$, so if $p>m$,$p=m\times (p-m) $  
	which contradicts the assumption that $p$ is a prime number. we can find out such verticle that remains same in every bijection and assume the number of them is $x$. For other verticles, since the grph is connected, there relative location must remain same in every bijection, and each verticle has only $(p-x)$ locations to be placed, so the number of automorphs is at most $p-x$, which contradicts the assumption. \par
	Assume that there is no such bijection, For other verticles, since the grph is connected, there relative location must remain same in every bijection, and each verticle has only $p$ locations to be placed, so the number of automorphs is at most $p$.\par
	To conclude, There is no such graph with less than $p$ verticles.	

\end{document}
	

