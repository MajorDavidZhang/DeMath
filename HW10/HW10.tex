\documentclass{article} % For LaTeX2e
\usepackage{nips15submit_e,times}
\usepackage[colorlinks,linkcolor=red]{hyperref}
\usepackage{url}
\usepackage{amsmath}
\usepackage{graphicx}
\usepackage{float}
\usepackage{bm}
\usepackage{amssymb}
%\documentstyle[nips14submit_09,times,art10]{article} % For LaTeX 2.09


\title{CS499 Homework 10 (First Draft)}


\author{
	Intersteller\thanks{ Use footnote for providing further information
		about author (webpage, alternative address)---\emph{not} for acknowledging
		funding agencies.}
	Department of Computer Science
	Cranberry-Lemon University
	Pittsburgh, PA 15213
}

% The \author macro works with any number of authors. There are two commands
% used to separate the names and addresses of multiple authors: \And and \AND.
%
% Using \And between authors leaves it to \LaTeX{} to determine where to break
% the lines. Using \AND forces a linebreak at that point. So, if \LaTeX{}
% puts 3 of 4 authors names on the first line, and the last on the second
% line, try using \AND instead of \And before the third author name.

\newcommand{\fix}{\marginpar{FIX}}
\newcommand{\new}{\marginpar{NEW}}

\newtheorem{theorem}{}

%\nipsfinalcopy % Uncomment for camera-ready version

\begin{document}
	
	
	\maketitle
	
	
	\textbf{Exercise 10.1}\par
	Since $$\sum_{v\in V}f(s,v)=\sum_{v\in V\backslash S}f(s,v)+\sum_{v\in S}f(s,v)$$ we only need to prove that $$\sum_{v\in S}f(s,v)=\sum_{u\in S-s,\ v\in V\backslash S}f(u,v)$$\\
	Since $$\sum_{v\in S}f(s,v)=-\sum_{v\in S-s}f(v,s)$$ we only need to prove that $$\sum_{u\in S-s,\ v\in s+V\backslash S}f(u,v)=0$$
	It is obvious to see that $$\sum_{u\in S-s,\ v\in S-s}f(u,v)=0$$
	So, we only need to prove that $$\sum_{u\in S-s,\ v\in s+V\backslash S}f(u,v)+\sum_{u\in S-s,\ v\in S-s}f(u,v)=\sum_{u\in S-s,\ v\in V}f(u,v)=0$$
	According to the defination, $$\sum_{u\in S-s,\ v\in V}f(u,v)=0$$
	Done.


	\textbf{Exercise 10.2}\par
	Define the minimum cut between i and j as minCut(i,j). According to the Max Flow Min Cut Theorem, minCut(s,r)$\geq$k, minCut(r,t)$\geq$k. Obviously, minCut(s,t)$\geq$min\{minCut(s,r),minCut(r,t)\}$\geq$k, which means there is a flow from s to r of value k.

	\textbf{Exercise 10.4}\par
	Suppose there is a s-t-path in G that has less that k edges. Then, at least one edge in the path moves more than one level forward, which contradicts Definition 10.3. So, $dist(s,t)\geq k$.

    \textbf{Exercise 10.5}\par
    Since dist$(s,t)=k$, we have a path $s\rightarrow u_1\rightarrow u_2 \cdots u_{k-1}\rightarrow t$. For convenience, we call $s:u_0,t:u_k$. We construct $V_0, V_1\cdots V_K$ as follows:\par
    Step$1$:$u_0\in V_0, u_1\in V_1, u_2\in V_2,\cdots, u_{k-1}\in V_{K-1}, u_k\in V_K.$\par
    Step$2$:$\forall v\in V \backslash V_k$, if there is an edge $(v,t)$, let $V_{k-1}$ contain $v$. Obviously, $\forall n, 0\leq n \leq k-2, u_n$ can't link to t. Otherwise, we can skip $u_{k-1}$ and get a shorter path from $s$ to $t$, which contradicts $dist(s,t)=k$.\par
    Step$3$:$\forall v\in V \backslash (V_k\cup V_{k-1})$, if $\exists edge (v,w), w\in V_{k-1}$, let $V_{k-2}$ contain $v$.
    $\forall n, 0\leq n\leq k-3,$ $u_n$ can't link to any vertex in $V_{k-1}.$\par
    \vdots
    Step$k$:$\forall v\in V \backslash \bigcup_{i=2}^kV_i$, if $\exists edge (v,w), w\in V_2$ , let $V_1$ contain $v$. Similarly, $u_0$ can't link to any vertex in $V_2$.\par
    Step$k+1$:$\forall v \in V $ and $ v \notin V_i, i \neq 1,2,3,\cdots, k, v \in V_0$.\par
    Therefore, if dist$(s,t)=k$, $(G,s,t,c)$ has a k-layering.

	
	\textbf{Exercise 10.6}\par
	Obviously, $(1)s\in V_{0}$ and $(2)t\in V_{k}$ are satisfied, we consider condition $(3)$. Since $(G,s,t,c)$ is a flow network 
	and $V_{0},V_{1},...,V_{k}$ is an optimal layering, every edge in $G$ moves at most one level forward and $dist_G(s,t)=k$. And we denote the path $p$ 
	as $s\rightarrow v_{1}\rightarrow v_{2} \rightarrow ...\rightarrow v_{k-1} \rightarrow t $. Then we can get  $s\in V_{0}$, $v_{1}\in V_{1}$, ... ,$v_{i}\in V_{i}$, ... ,$v_{k-1}\in V_{k-1}$ and $t\in V_{k}$. Each edge is between two adjacent layerings. Since residual network $G_{f}$ only add a reverse edge of each edge in $p$, these additional edges are also between two adjacent layerings. So $V_{0},V_{1},...,V_{k}$ satisfy condition $(3)$ and it is a layering of $(G_{f},s,t,c_{f})$.\par

    \textbf{Exercise 10.7}\par
    Because there is a path from $s$ to $t$, we suppose $dist_G(s,t)=k$. Based on the construction method in $10.5$, we can find a k-layering and it is an optimal layering. Therefore, every network$(G,s,t,c)$ has an optimal layering, provided there is a path from $s$ to $t$.
	
	\textbf{Exercise 10.8}\par
	We consider each while-loop of EK algorithm. In every iteration, EK algorithm choose $p$ to be a shortest s-t-path in $G_{f}$. 
	And we denote the path $p$ as $v_{0}\rightarrow v_{1}\rightarrow v_{2} \rightarrow ...\rightarrow v_{k-1} \rightarrow v_{k} $ and $v_{0}=s\in V_{0}$, $v_{1}\in V_{1}$, ... ,$v_{i}\in V_{i}$, ... ,$v_{k-1}\in V_{k-1}$ and $v_{k}=t\in V_{k}$. Then, EK algorithm routes $c_{min}$ flow along $p$. In $G$, $\exists (v_{i},v_{i+1})\in p$ where $i\in {0,1,2...,k}$, $c_{f}(v_{i},v_{i+1})=c_{min}$. In $G_{f}$, $c_{f}'(v_{i},v_{i+1})=0$ and  $c_{f}'(v_{i+1},v_{i})=c_{min}$. \\
	Obviously, if  $V_{0},V_{1},...,V_{k}$ is still an optimal layering, $c_{f}(v_{i},v_{i+1})$ is always $0$. Otherwise, there is a feasible s-t-path from $v_{i+1}$ to $v_{i}$, which is impossible. Therefore, in every iteration, the total number of the edges, which are from $V_{i}$ to $V_{i+1} (\forall i \in{0,1,2,..,k})$ and in feasible s-t-path($dist(s,t)=k$), will minus at least one. Obviously, these edges are less than or equal to $m$. So, after $m$ iteration, there is no feasible s-t-path which $dist(s,t)=k$, and $dist(s,t)$ will be larger than $k$. So, after at most $m$ iterations of  the while-loop, $V_{0},V_{1},...,V_{k}$ ceases to be an optimal layering.\par
	
	\textbf{Exercise 10.9}\par
	According to \textbf{Exercise 10.8}, a particular layering is no more optimal after at most $m$ iterations. Since a layering is at least 1-layering and at most n-latering, after at most $m*n$ iterations, there is no optimal layering, which means there is no s-t-path, the algorithm terminates.
	
	
	\textbf{Exercise 10.10}\par
	According to \textbf{Exercise 10.9},  the Edmonds-Karp algorithm terminates after n·m iterations of the while-loop, which is to say, we can get the max flow $f$ after finite steps by Edmonds-Karp algorithm.

		\textbf{Question}\par
	Can we use the Edmonds-Karp algorithm in multi-source network?
	
\end{document}

