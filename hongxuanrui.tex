\documentclass{article} % For LaTeX2e
\usepackage{nips15submit_e,times}
\usepackage[colorlinks,linkcolor=red]{hyperref}
\usepackage{url}
\usepackage{amsmath}
\usepackage{graphicx}
\usepackage{float}
\usepackage{bm}
\usepackage{amssymb}
%\documentstyle[nips14submit_09,times,art10]{article} % For LaTeX 2.09


\title{CS499 Homework 3 (First Draft)}


\author{
	Intersteller\thanks{ Use footnote for providing further information
		about author (webpage, alternative address)---\emph{not} for acknowledging
		funding agencies.}
	Department of Computer Science
	Cranberry-Lemon University
	Pittsburgh, PA 15213
}

% The \author macro works with any number of authors. There are two commands
% used to separate the names and addresses of multiple authors: \And and \AND.
%
% Using \And between authors leaves it to \LaTeX{} to determine where to break
% the lines. Using \AND forces a linebreak at that point. So, if \LaTeX{}
% puts 3 of 4 authors names on the first line, and the last on the second
% line, try using \AND instead of \And before the third author name.

\newcommand{\fix}{\marginpar{FIX}}
\newcommand{\new}{\marginpar{NEW}}

\newtheorem{theorem}{}

%\nipsfinalcopy % Uncomment for camera-ready version

\begin{document}
	
	
	\maketitle
	
	
	\textbf{Exercise 7.8}\par
    As the picture shows , for $\forall c,m_c(T)=m_c(T\prime)$.

  

	\textbf{Exercise 7.9}\par
    Sort by the weight of $T's$ edges and $T\prime's$ edges , we have $(a1,a2,a3,\cdots,a_{n-1})$ , $(b1,b2,b3,\cdots,b_{n-1})$ . Suppose $a_i\ne b_i$ , $\forall k<i,a_k=b_k$ and $w(a_i)\geq w(b_i)$ , there are two situations:\par
    $(1)$ edge $b_i$ exists in the $T$ , then we can find $j(j>i)$ and $a_j=b_i$ . Because $w(b_i)=w(a_j)\geq w(a_i)\geq w(b_i) , w(a_i)=w(b_i)=w(a_j) . $So we can exchange $a_i$ and $a_j$ and new sequence is still ordered . $T's$ and $T\prime's$ i position is the same edge.\par
    $(2)$ edge $b_i$ doesn't exist in the $T$ , then we add $b_i$ to $T$ to form a cycle . Because $T$ is a minimum spanning tree , $w$(edge in the cycle)$\leq w(b_i)$ . And we can find $a_j$($j>i$ and $a_j$ doesn't exist in the $T\prime$ and $a_j$ in the cycle) . Because $w(b_i)\geq w(a_j)\geq w(a_i) \geq w(b_i),w(b_i)=w(a_i)=w(a_j).$ So we can change $a_j$ with $b_i$ . Turn to the situation $(1)$.\par
    So we know the ordered edge weight list of any two minimum spanning trees is the same.\par
    Obviously, $m_c(T)=m_c(T\prime).$

	
	\textbf{Exercise 7.10}\par
	Suppose there are two minimum spanning tree , sort by the weight of $T's$ edges and $T\prime's$ edges , we have$(a1,a2,a3,\cdots,a_{n-1})$ , $(b1,b2,b3,\cdots,b_{n-1})$.$\exists i,a_i \ne b_i$,based on the $7.9$,the ordered edge weight list of any two minimum spanning trees is the same,so $w(a_i)=w(b_i)$.But no two edges of $G$ have the same weight , so there is contradiction . So $G$ has exactly one minimum spanning tree!





	\textbf{Exercise 2.4}\par
	
\end{document}

